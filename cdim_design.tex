%%%%%%%%%%%%%%%%%%%%%%%%%%%%%%%%%%%%%%%%%%%%%%%%%%%%%%%%%%%%%%%%%%%%%%%%%%%%
%% Trim Size : 11in x 8.5in
%% Text Area : 9.6in (include Runningheads) x 7in
%% ws-jai.tex, 26 April 2012
%% Tex file to use with ws-jai.cls written in Latex2E.
%% The content, structure, format and layout of this style file is the
%% property of World Scientific Publishing Co. Pte. Ltd.
%%%%%%%%%%%%%%%%%%%%%%%%%%%%%%%%%%%%%%%%%%%%%%%%%%%%%%%%%%%%%%%%%%%%%%%%%%%%
%%

%\documentclass[draft]{ws-jai}
\documentclass{ws-jai}
\usepackage[flushleft]{threeparttable}
\usepackage{siunitx}
\usepackage{booktabs}
\usepackage[T1]{fontenc}
\hyphenpenalty=2000
\bibliographystyle{ws-jai}
\begin{document}

\catchline{}{}{}{}{} % Publisher's Area please ignore

\markboth{P.~Linden \textit{et al.}}{CDIM:\@ Probe Class Space Telescope Design}

\title{Cosmic Dawn Intensity Mapper: \\Spacecraft and Mission Design for a Probe-Class Space Telescope}

\author{Philip Linden$^{1,\dagger}$, Michael Zemcov$^{2}$}

\address{
$^1$Department of Mechanical Engineering, Kate Gleason College of Engineering, Rochester
Institute of Technology, Rochester, NY 14623, USA, pjl7651@rit.edu\\
$^2$Center for Detectors, School of Physics and Astronomy, Rochester
Institute of Technology, Rochester, NY 14623, USA, zemcov@cfd.rit.edu\\
}

\maketitle

\corres{$^\dagger$Corresponding author.}

\begin{history}
\received{(to be inserted by publisher)};
\revised{(to be inserted by publisher)};
\accepted{(to be inserted by publisher)};
\end{history}

\begin{abstract}
  Very abstract.\\
  Much interesting.
\end{abstract}

\keywords{A list of 3--5 keywords are to be supplied, separated by commas.}

\section{Introduction}
\label{S:introduction}
% Discuss the relevance to NASA, including the science objectives of the mission and how the mission satisfies a Probe class mission.
% http://sites.nationalacademies.org/cs/groups/bpasite/documents/webpage/bpa_064932.pdf
Observing the behavior and characteristics of the earliest stars and galaxies is fundamental to understanding the physics that led to their formation and evolution.
Breakthrough discoveries in understanding the physics of the epoch of reionization are anticipated in the 2020--2030 decade thanks to WFIRST and JWST\@.
However, JWST's capability will be limited to only several cosmological deep fields.
Although WFIRST will be capable of wide area surveys, its spectroscopy is limited to \SI{2}{\micro\meter} and thus limits the selection of galaxies it is able to observe.
Neither JWST nor WFIRST provide a complete understanding of the epoch of reionization, specifically in terms of answering the questions of when and how the cosmic dawn came to be.
This area of research is a prime candidate for a Probe class mission optimized to study reionization.

% CDIM fills a gap in the 2020 Decadal between two other not so famous telescopes in the 2030 range after JWST and WFIRST and Hubble

% What is a "Probe class" mission?
Probe class missions occupy a role on a larger scale than Discovery missions, such as Kepler and Dawn, but not as vast as Flagship missions such as JWST~\cite{probeclasswp}.
Such missions are intended to be PI-led scientific investigations rather than general observatories, and have a firm \$1B cap.

% % How is the design of the spacecraft is driven by the science objectives?
% \emph{Straight from Asantha's paper:} \\
% Cosmic Dawn Intensity Mapper (CDIM) is 1.5m-class infrared telescope capable of three-dimensional spectro-imaging observations over the wavelength range of 0.75 to \SI{7.5}{\micro\meter}, at a spectral resolving power \si{\Delta\lambda\per\lambda} of 500.
% This will be achieved with linear variable filters (LVFs) sitting on top of a focal plane of thirty-six 2048$\^{2}$ detectors.
% The field-of-view (FoV) of CDIM will be \SI{10}{\deg\squared} instantaneously. The survey strategy using spacecraft operations following a shift and stare mode will result in 1360 independent narrow-band spectral images of the sky on a given location.
% Currently prioritized science programs, taking over three-years of a five-year mission, will be accomplished with a two-tiered wedding-cake survey with the shallowest spanning close to \SI{300}{\deg\squared} and the deepest tier of about \SI{25}{\deg\squared}.
% The remaining two-years could be used for additional survey programs (the wide tier can be expanded to \SI{1000}{\deg\squared}) or for use by the astronomical community through a General Observing (GO) campaign.
% \emph{END ASANTHA}
Cosmic Dawn Intensity Mapper (CDIM) is a Probe class infrared telescope capable of three-dimensional spectro-imaging observations over the wavelength range of 0.75 to \SI{7.5}{\micro\meter}, at a spectral resolving power \si{\Delta\lambda\per\lambda} of 500.
CDIM has a \SI{10}{\deg\squared} instantaneous field of view (FoV) utilizing linear variable filters (LVFs) atop a focal plane of thirty-six $2048\times2048$ detectors.
The survey strategy using spacecraft operations following a shift and stare mode will result in 1360 independent narrow-band spectral images of the sky on a given location.
Surveys could span from \SI{25}{\deg\squared} up to \SI{1000}{\deg\squared} over a five year lifetime in an orbit about Lagrange point L$_{2}$\@.

With these instrument requirements, CDIM is optimized to search for the first cosmic sources of dust and evidence of the very earliest stellar populations, bridging the gaps in JWST and WFIRST cosmic dawn surveys and exceeding them in capability.

CDIM is notionally a \SI{1.5}{\meter} aperture telescope, passively cooled to \SI{45}{\kelvin}, with a $6\times6$ detector array that utilizes linear variable filters (LVFs) and actively cooled to \SI{35}{\kelvin}~\cite{cooray2016cdim2page}.
% Key technical characteristics of CDIM are summarized in Table~\ref{table:techreqs}.

% FIGURE: Summarize the requirements of a spacecraft that should make these observations, including the spectral and spatial characteristics, sensitivity, and logistics (\textit{i.e.} location, lifetime).
% \begin{table}[ht]
% \centering
% \caption{CDIM Instrument Technical Requirements}
% \bigskip
%   \begin{tabular}{lr}
%     \toprule
%     \textbf{Parameter} & \textbf{Requirement} \\
%     \midrule
%     Wavelength range                                & 0.75 to \SI{7.5}{\micro\meter} \\
%     Spatial resolution at \SI{5}{\micro\meter}      & \theta_{pix} \textless{} \SI{2}{\arcsecond} \\
%     Photometric sensitivity at \SI{5}{\micro\meter} & 27 AB mag \\
%     Spectral line sensitivity & \si{4\times10^{-18}} ergs s^{-1} \si{\centi\meter^{-2}} \\
%     Spectral resolving power & \si{\Delta\lambda\per\lambda} \textgreater{} 300 \\
%     Surface brightness sensitivity & (?) \\
%     \bottomrule
%   \end{tabular}
% \label{table:techreqs}
% \end{table}

% FIGURE: Interdependency chart?

\section{Mirror}
\label{S:mirror}
% Recall requirements such as FoV, spectral resolution. Technical requirement of temperature.
Preliminary explorations indicate that a \SI{1.5}{\meter} primary mirror cooled to \SI{45}{\kelvin} is required to meet CDIM's spectro-imaging requirements~\cite{cooray2016cdim2page}.
Such a mirror would be be fabricated from aluminum with a gold-deposition surface finish.
``Back-of-the-napkin'' calculations estimate the mirror's mass to be around \SI{50}{\kilo\gram}.

\section{Detector}
\label{S:detector}
Recall requirements of spatial resolution, wavelength range, and sensitivity. Technical requirement of temperature. Explain detector type and pixel size that fulfils these.

Multiple detectors are satisfactory, range in TRL, but all are pretty far in development and have been demonstrated in NEOCam, SPHEREx, and JWST\@.

Using H2RG, need this many which results in this power draw.

\section{Thermal Regulation}
\label{S:tempregulation}
General approach to cooling with passive and active. Explain tradeoffs between passive and active.

\subsection{Passive Cooling}
\label{sS:vgrooves}
Explain generally how v-groove radiators work.
Used on planck and jwst.
Desired final temp based on material, setup and location.
Area based on general size of telescope.
May need to be deployable, depending on required area.

\subsection{Active Cooling}
\label{sS:cryocooler}
Detectors use active cooling to bring temp down to \SI{35}{\kelvin}.
Explain why active needed to manage thermals of detectors.
Target temperature and estimated heat dissipation.

Describe types of cryocoolers and the high-level traits/tradeoffs between them. Choose one in particular, but leave wiggle room for others to be chosen.
Explain the architecture for implementing this type of cooler, including power draw and mass.
Active cooling will be achieved by a pulse-tube or stirling cycle mechanical cryocooler.
Typical input power to a \SI{5}{\watt} cryocooler is around \SI{70}{\watt} at close to \SI{10}{\kilo\gram}.

\section{Spacecraft Bus}
\label{S:bus}
The spacecraft/vehicle/bus houses all non-instrumentation systems including the cryocooler, ADCS, and power modules.
It would have approximately some dimensions and be made of standard materials.
The bus will also include hard points for integration with the launch vehicle.

\subsection{Attitude Determination and Control}
\label{sS:adcs}
Recall attitude control requirements for science objectives.
System architecture requires the spacecraft to first understand its orientation and then act to align itself with a given area of the sky.

Explain the star tracker among other attitude control systems.
Select a class of star tracker.

Briefly explain inertial and propulsive attitude control and the limitations of each.
The system will ideally be inertial only, but larger systems like JWST need a combination of both.

Attitude determination and control systems account for a percentage of the total power draw of the system.

\subsection{Telemetry}
\label{sS:telemetry}
The spacecraft must receive commands from ground stations at Earth and transmit telemetry and data from L2.
Since the desired data transfer rate is what it is, this type of telemetry system is required.

\subsection{Power}
\label{sS:power}
All power generation will come from an array of photovoltaic cells facing the sun.
In order to survey the entire sky, the cells must be able to adjust to account for different incident angles to the sun.
The array must deploy after the launch portion of the mission.

Based on a rough power budget and the spacecraft's position at L2, the array must be a bunch of square meters in area to sustain operation.
The dark side of the array acts as a radiator to contribute to the thermal regulation of the spacecraft bus.

Battery banks will store energy.

\section{Mission Profile}
\label{S:mission}
The mission profile encompasses spacecraft and ground operations from launch until the end of the scientific mission (5 years).

\subsection{Launch Phase}
\label{sS:launch}
The spacecraft will be integrated to a launch vehicle such as an Ariane V (which will deliver JWST) or Ares V (9400kg capacity to L2) and delivered to L2.
Launch vehicle selection is limited by spacecraft mass and size.

\subsection{Initialization Phase}
\label{sS:initphase}
The deployables deploy and the v-grooves begin to cool the spacecraft down to the passive threshold.
The response time of the passive coolers is very long as mentioned above.
After the initial cooldown, active cooling engages.

\subsection{Observation Phase}
\label{sS:obsphase}
The spacecraft is ready to perform observations and survey the early universe.
This is the nominal operation phase and will last five years.

\section{Cost Estimations}
\label{S:cost}
The overall cost of a space telescope may be broken down into a set of drivers whose influence may be correlated with historical data.
All conclusions based on statistical analysis are only as good as their databases, and fiscal data, such as what is required for proper analysis, is scarce.
Estimations are made with engineering judgement based on available data. Multiple models will be used to develop a reasonable cost model for CDIM.\@

H. Philip Stahl presents a statistical approach to estimating OTA cost based on correlations with data on flown space telescope missions [1].
CDIM's projected costs may be obtained from Stahl's findings with engineering judgment, knowing that the data is drawn from a relatively small sample set.

\subsection{Relative Influences on Total Cost}
Two leaders in optical telescope design, Pierre Bely and H. Philip Stahl, developed cost breakdowns for a typical space telescope.
Bely's slightly older model is shown in Figure 1 [2] and Stahl's model is outlined in Figure 2 [1].
Discrepancies may be attributed to differing source data sets (among other things) but in general the two agree with one another.

Neither Stahl nor Bely's cost models, upon which this report heavily relies, includes NASA labor or overhead [2, pp. 073006--2].
Thus, an additional factor of 0.5 shall be included in the Total Cost to account for NASA overhead and labor, where:

\begin{equation}
  	$Total Cost$=$Mission Cost$ (1+$Factor$)
\end{equation}

Given a baseline estimate of \$850M total mission cost, Table 1 summarizes a range of target costs may be estimated from Bely and Stahl in Figure 1 and Figure 2, adjusted to account for labor and overhead.

\begin{figure}
  \caption{BELY'S TYPICAL TELESCOPE COST BREAKDOWN IN PERCENT OF MISSION COST (2002) [2]
\label{fig:bely-cost}
}
\end{figure}

\begin{figure}
  \caption{STAHL'S TYPICAL SPACE TELESCOPE COST BREAKDOWN IN PERCENT OF MISSION COST (2011) [1]
\label{fig:stahl-mission-cost}
}
\end{figure}

\begin{table}
  \caption{FIRST ORDER COST ESTIMATION BASED ON \$850M TOTAL MISSION COST, INCLUDING OVERHEAD
\label{tab:first-order-cost-estimate}
}
\end{table}

Applying this logic we may develop an estimate for CDIM that includes significant factors on budget like the Science Team and Ground Support.
A possible cost breakdown is presented in Figure 3 and Table 2.

\begin{figure}
  \caption{PROPOSED COST BREAKDOWN FOR CDIM, NOT INCLUDING LABOR OR OVERHEAD
\label{fig:cost-no-overhead}
}
\end{figure}

\begin{table}
  \caption{PROPOSED COST BREAKDOWN FOR CDIM BASED ON \$850M TOTAL MISSION COST
\label{tab:cost-from-total-mission}
}
\end{table}

\subsection{OTA Cost Estimation}
Spitzer is a representative benchmark for CDIM since it is similar in cost, diameter, and operational wavelength, summarized in Table 3.
Additionally, Spitzer resides on the median curve in Stahl's cost model shown in Figure 4.

\begin{figure}
  \caption{STAHL OTA COST VS DIAMETER CORRELATION (2011) [1]
\label{fig:cost-stahl-ota-cost-vs-diameter}
}
\end{figure}

\begin{table}
  \caption{SPITZER VS CDIM WITH OTA COST ESTIMATES BASED ON STAHL MODEL
\label{tab:cost-spitzer-vs-CDIM}
}
\end{table}

The Stahl Model for OTA Diameter vs. OTA Cost in Figure 4 estimates an OTA mirror aperture diameter of 1.3 meters costing around \$42M, which is comfortably within the OTA budget of \$62--85M from the Bely and Stahl Mission Cost Models in Table 1.

Pushing the boundaries of these cost models, the median in Figure 4 places a 1.6 meter OTA aperture diameter (~\$62.6M) within range of the Bely budget of \$62.3M, or a 1.9 meter OTA aperture diameter (~\$84.3M) within the Stahl budget of \$85M.
These estimates are summarized in Table 4.

\begin{table}
  \caption{OTA COST VERSUS DIAMETER BASED ON MEDIAN OF STAHL MODEL
\label{tab:cost-ota-vs-dia-CDIM}
}
\end{table}

\subsection{Using OTA Cost to Determine Total Cost}
Choosing a 1.3m telescope at \$48.0M as the driving factor on cost, alternative Mission Cost and Total Cost estimates may be derived from Stahl and Bely models as shown in Table 5.
There is a significant (\$200M+) discrepancy between the Bely and Stahl cost models, but both estimations show that CDIM fits squarely within the range of a Probe class NASA mission.

\begin{table}
  \caption{TOTAL COST ESTIMATED FOR 1.3M OTA
\label{tab:cost-tota-1.3m}
}
\end{table}

Figure 5 shows other mission and total cost estimates generated by iterating on this process for other OTA diameters listed in Table 4.

\begin{figure}
  \caption{TOTAL COST RELATION TO OTA DIAMETER UNDER EACH COST MODEL
\label{fig:cost-total-compare-models}
}
\end{figure}

The Stahl OTA Diameter vs. OTA Cost model is based on a limited set of data and is not a perfect correlation, and does not account for influences that other parameters like operating temperature have on Mission Cost.
It does show a reasonable correlation with historical data, however, and may be used for preliminary design phases for missions such as CDIM.\@

My recommendation is to develop a conservative design around a 1.3m (\$42M) maximum diameter OTA, and allow the remaining funds in a \$850M preliminary budget to be spent on science investigations, instruments, and spacecraft components such as a thermal regulation system.

\section{Conclusion}
\label{S:conclusion}
Summarize points made above. Notable points include mirror diameter, detector, power budget, mass budget, total cost, and the fact that technologies are already significantly developed.

\textbf{FIGURE} Render of spacecraft with bus?

\section*{Acknowledgments}
Thanks.

\bibliography{cdim_design}

\end{document}
