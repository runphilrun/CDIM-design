%%%%%%%%%%%%%%%%%%%%%%%%%%%%%%%%%%%%%%%%%%%%%%%%%%%%%%%%%%%%%%%%%%%%%%%%%%%%
%% Trim Size : 11in x 8.5in
%% Text Area : 9.6in (include Runningheads) x 7in
%% ws-jai.tex, 26 April 2012
%% Tex file to use with ws-jai.cls written in Latex2E.
%% The content, structure, format and layout of this style file is the
%% property of World Scientific Publishing Co. Pte. Ltd.
%%%%%%%%%%%%%%%%%%%%%%%%%%%%%%%%%%%%%%%%%%%%%%%%%%%%%%%%%%%%%%%%%%%%%%%%%%%%
%%

%\documentclass[draft]{ws-jai}
\documentclass{ws-jai}
\usepackage[flushleft]{threeparttable}
\bibliographystyle{ws-jai}
\begin{document}

\catchline{}{}{}{}{} % Publisher's Area please ignore

\markboth{P.~Linden \textit{et al.}}{CDIM Spacecraft Design}

\title{Cosmic Dawn Intensity Mapper: \\Spacecraft and Mission Design for a Probe-Class Space Telescope}

\author{Philip Linden$^{1,\dagger}$, Michael Zemcov$^{2}$}

\address{
$^1$Department of Mechanical Engineering, Kate Gleason College of Engineering, Rochester
Institute of Technology, Rochester, NY 14623, USA, pjl7651@rit.edu\\
$^1$Center for Detectors, School of Physics and Astronomy, Rochester
Institute of Technology, Rochester, NY 14623, USA, zemcov@cfd.rit.edu\\
}

\maketitle

\corres{$^\dagger$Corresponding author.}

\begin{history}
\received{(to be inserted by publisher)};
\revised{(to be inserted by publisher)};
\accepted{(to be inserted by publisher)};
\end{history}

\begin{abstract}
  Very abstract.
  Much interesting.
\end{abstract}

\keywords{A list of 3--5 keywords are to be supplied, separated by commas.}

\section{Introduction}
\label{S:introduction}
The introduction goes here.
%
Discuss the relevance to NASA, including the science objectives of the mission and how the mission satisfies a Probe class mission.
%
Describe the scope of the paper, including how the design of the spacecraft is driven by the science objectives.

\subsection{Requirements}
\label{sS:requirements}
Also explain the requirements of a spacecraft that should make these observations, including the spectral and spatial characteristics, sensitivity, and logistics (\textit{i.e.} location, lifetime).

\section{Mirror}
Recall requirements such as FoV, spectral resolution. Technical requirement of temperature.
Based on spectral resolution and FoV, need between 1m-1.5m. Based on cost estimates from Stahl, estimate target diameter, which rersults in target mass.

\section{Detector}
Recall requirements of spatial resolution, wavelength range, and sensitivity. Technical requirement of temperature. Explain detector type and pixel size that fulfils these.

Multiple detectors are satisfactory, range in TRL, but all are pretty far in development and have been demonstrated in NEOCam, SPHEREx, and JWST\@.

Using H2RG, need this many which results in this power draw.

\section{Thermal Regulation}
General approach to cooling with passive and active. Explain tradeoffs between passive and active.

\subsection{Passive Cooling}
Explain generally how v-groove radiators work. Used on planck and jwst. Desired final temp based on material, setup and location. Area based on general size of telescope. May need to be deployable.

\subsection{Active Cooling}
\textbf{NEEDS RESEARCH}\\
Detectors use active cooling to bring temp below passive temp. Explain why active needed to manage thermals of detectors. Target temperature and estimated heat dissipation.

Describe types of cryocoolers and the high-level traits/tradeoffs between them. Choose one in particular, but leave wiggle room for others to be chosen. Explain the architecture for implementing this type of cooler, including power draw and mass.

\section{Attitude Determination and Control}
\textbf{NEEDS RESEARCH}\\
Recall attitude control requirements for science objectives. System architecture requires the spacecraft to first understand its orientation and then act to align itself with a given area of the sky.

\subsection{Determination Systems}
Explain the star tracker among other attitude control systems. Select a class of star tracker.

\subsection{Control Systems}
Briefly explain inertial and propulsive attitude control and the limitations of each. The system will ideally be inertial only, but larger systems like JWST needa combination of both

\section*{Acknowledgments}
Thanks.

\bibliography{cdim_design}

\end{document}
